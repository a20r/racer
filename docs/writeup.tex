
\title{Formalization of Methods}

\date{\today}

\author{Alex Wallar}

\documentclass{article}

\usepackage{amsfonts}

\usepackage{amsmath}

\usepackage[pdftex]{graphicx}

\usepackage{relsize}

\usepackage[margin=1.4in]{geometry}

\usepackage{float}

\usepackage{fancyvrb}

\usepackage{algorithm}

\usepackage[noend]{algorithmic}

\floatstyle{ruled} \newfloat{program}{thp}{lop} \floatname{program}{Structure}

\newcommand{\Normal}[2]{\mathcal{N}_k(#1, #2)}

\newcommand{\Acronym}[1]{\ensuremath{{\small{\texttt{#1}}}}}
\newcommand{\Constant}[1]{\ensuremath{\small{\texttt{#1}}}}
\newcommand{\Rover}{\Acronym{Rover}}
\newcommand{\Keeper}{\Acronym{Keeper}}
\newcommand{\False}{\Constant{false}}
\newcommand{\True}{\Constant{true}}
\newcommand{\Symbol}[1]{\ensuremath{\mathcal{#1}}}
\newcommand{\Function}[1]{\ensuremath{{\small \textsc{#1}}}}
\newcommand{\Var}[1]{\ensuremath{{\small{\textsl{#1}}}}}
\newcommand{\argmax}[1]{\underset{#1}{\operatorname{arg}\,\operatorname{max}}\;}
\newcommand{\Max}[1]{\underset{#1}{\operatorname{max}}\;}
\newcommand{\grad}[1]{\underset{#1}{\operatorname{\Function{GradientDecent}}}\;}
\newcommand{\s}{\textbf{s}}
\newcommand{\fig}[1]{\textbf{Fig.~\ref{fig:#1}}}

\newcommand{\n}[2]{\ensuremath{n^{(#2)}_{#1}}}

\newcommand{\PA}{\ensuremath{P^{(\n{i}{t},\;\n{j}{t})}_{A}}}
\newcommand{\PD}{\ensuremath{P^{(t_0,\;t_m)}_{a}}}
\newcommand{\PAll}{\ensuremath{P^{(t_0,\;t_m)}_{A}}}

\begin{document}

\maketitle

\section{Temporal Probabilistic Roadmap}

\paragraph{Construction} A temporal probabilistic roadmap (tPRM) is similar to
other probabilistic roadmaps but instead of just representing the configuration
space of the robot, a tPRM also incorporates the time at which the robot would
be at that configuration. For instance, in a two dimensional setting, the nodes
of the tPRM would be of the form $(x, y, t)$ where $t$ is the time at which the
robot would be at $(x, y)$. The addition of time into the tradition approach
allows more constraints to be set on the construction of the roadmap that
include time. The tPRM is represented by a weighted directed graph, $(V, E,
W)$, such that an edge can only exist from $n_i$ to $n_j$ if $\n{i}{t} <
\n{j}{t}$ along with other constraints set during the construction.  The
temporal probabilistic roadmap is constructed by sampling random points within
the search space and connecting nodes that are feasible for the robot to move
between.  More formally, an edge exists between $n_i$ and $n_j$ if $F_R(n_i,
n_j) = 1$ where the function, $F_R: V \times V \to \{0, 1\}$ represents whether
or not it is feasible to move between two vertices in the roadmap for a given
robot model, $R$.  Other than the feasibility constraint for an edge between
two nodes, there are also constraints that limit the size of edges. We assume
that there is a maximum time $T_m > 0$ such that if $(n_i, n_j) \in E$, then
$\n{j}{t} - \n{i}{t} \leq T_m$. Likewise, we assume that there is a maximum
distance that can exist between nodes, $D_m > 0$, such that if $(n_i, n_j) \in
E$, then $\Var{dist}(n_i, n_j) \leq D_m$ where $\Var{dist}(n_i, n_j)$ is the
distance between $n_i$ and $n_j$ in Euclidean space. Algo.~\ref{algo:tprm}
shows pseudocode for the construction of the tPRM.

\paragraph{Search} In order to get a set of sub-goals that can be used to lead
the robot to the goal, a graph search algorithm is used. The current
implementation uses Dijkstra's shortest path algorithm because it is provided
in the \verb|NetworkX| graph library by default. Given the constructed roadmap,
the algorithm is used to find the shortest path in the directed graph from the
robot's initial configuration to it's goal configuration.

\begin{algorithm}[ht]

\caption{Pseudocode for tPRM construction}

\label{algo:tprm}

\begin{algorithmic}[1]

\setcounter{ALC@line}{0}

\vspace*{1mm}

\STATE $G \leftarrow \Function{InitDirectedGraph}()$

\STATE $S \leftarrow \{\Var{start}\}, i \leftarrow 0$

\WHILE{$i < \Var{MaxPoints}$}

\STATE $i \leftarrow i + 1$

\STATE $\rho \leftarrow \Function{RandomChoice}(S)$

\STATE $\sigma \leftarrow \Function{Sample}(\rho, F_R)$

\IF{$\Function{WithinSearchSpace}(\sigma)$}

\STATE $S \leftarrow S \cup \{\sigma\}$

\ENDIF

\FORALL{$s \in S$}

\IF{$\Var{dist}(s, \sigma) \leq D_m \wedge |\sigma^{(t)} - s^{(t)}| \leq
T_m$}

\IF{$s^{(t)} < \sigma^{(t)} \wedge F_R(s, \sigma) = 1$}

\STATE $\Function{AddEdge}(G, s, \sigma, P^{(s^{(t)},\;\sigma^{(t)})}_{A}(s,
\sigma))$

\ENDIF

\IF{$s^{(t)} > \sigma^{(t)} \wedge F_R(\sigma, s) = 1$}

\STATE $\Function{AddEdge}(G, \sigma, s, P^{(\sigma^{(t)},\;\s^{(t)})}_{A}(
\sigma, s))$

\ENDIF

\ENDIF

\ENDFOR

\ENDWHILE

\RETURN $G$

\end{algorithmic}
\end{algorithm}

\section{Agent Position Probability Distribution}

For a given set of agents, $A$, in the environment and their associated
trajectories, $\{\zeta_a : a \in A\}$, where $\zeta_a: \mathbb{R} \to
\mathbb{R}^k$, the probability of an agent being given position is represented
between a given time interval is given by the function, $\PD: \mathbb{R}^k \to
(0, 1)$, where $k$ is the dimension of the search space. $\PD$ is defined as,

$$\PD(\mathbf{x}^k) = \mathlarger{\int_{t_0}^{t_m}}\Normal{\zeta_a(t)}{(t_m -
t)^2 \cdot \alpha}(\mathbf{x}^k)dt$$

Where $\mathbf{x}^k$ represents the $k$-dimensional input vector, $\alpha$ is a
scaling constant, and $\mathcal{N}_k(\mu, \sigma^2)$ is a $k$-dimensional
Gaussian distribution centered at $\mu$ with a variance $\sigma^2$. For all of
the agents, the probability density function becomes

$$\PAll(\mathbf{x}^k) = \frac{\mathlarger{\sum_{a \in A}}
\PD(\mathbf{x}^k)}{|A|}$$

\section{Edge Weights}

For two nodes, $n_i$ and $n_j$, in the temporal probabilistic roadmap (tPRM),
where $\n{i}{t} < \n{j}{t}$, where $n^{(t)}$ represents the time at which robot
would be at node $n$, the weight of the edge between those two nodes is,

$$w(n_i, n_j) = \gamma \cdot \mathlarger{\max_{t \in [\n{i}{t},
    \n{j}{t}]}}(\PA \circ E)(t)$$

Where $E(t)$ is a function, $E: \mathbb{R} \to \mathbb{R}^k$, that represents
the position along the edge between $n_i$ and $n_j$ given a time, $t
\in [\n{i}{t},\;\n{j}{t}]$ and $k$ is the dimension of the search space. In the
equation, $\circ$ represents the function composition, and $\gamma$ is a
scaling constant.

\end{document}
