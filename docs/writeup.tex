
\title{Generating Safe Paths in Cluttered Dynamic
Environments by Predicting the Movements of Obstacles}

\date{\today}

\author{120013913}

\documentclass{article}
\usepackage{amsmath}
\usepackage{enumerate}
\usepackage{amssymb}
\usepackage[pdftex]{graphicx}
% \usepackage{relsize}
\usepackage{float}
\usepackage{algorithm}
\usepackage{fancyvrb}
\usepackage[noend]{algorithmic}

\usepackage{url}

\usepackage[margin=1.570796in]{geometry}

\floatstyle{ruled} \newfloat{program}{thp}{lop} \floatname{program}{Structure}
\newcommand{\Normal}[3]{\mathcal{N}(#1, #2, #3)}
\newcommand{\Acronym}[1]{\ensuremath{{\small{\texttt{#1}}}}}
\newcommand{\True}{\Constant{true}}
\newcommand{\Symbol}[1]{\ensuremath{\mathcal{#1}}}
\newcommand{\Function}[1]{\ensuremath{{\small \textsc{#1}}}}
\newcommand{\Constant}[1]{\ensuremath{\small{\texttt{#1}}}}
\newcommand{\Var}[1]{\ensuremath{{\small{\textsl{#1}}}}}
\newcommand{\argmin}[1]{\underset{#1}{\operatorname{arg}\,\operatorname{min}}\;}
\newcommand{\grad}[1]{\underset{#1}{\operatorname{\Function{GradientDecent}}}\;}

\begin{document}

% \maketitle

\section{Problem Description}

A pivotal problem in robotics is to generate collision free paths through
environment that allow a robot to move from an initial configuration to a goal
configuration without colliding with any obstacle. In many cases, these
environments will have dynamic obstacles that do not stay in the same position.
Using machine learning algorithms, it is possible to predict the future
positions of the obstacles by observing the obstacles' past positions. The
machine learning algorithms provide a probability distribution of where
obstacles are going to be within a certain amount of time into the future.  The
amount of time within the future is provided as an argument to the machine
learning algorithm. More formally, the machine learning algorithm,

$$P: \mathcal{O} \times \mathbb{R} \rightarrow (\mathbb{R}^2 \rightarrow [0,
1])$$

Where $\mathcal{O}$ is a set of all the obstacles and their past positions.
What is returned from $P$ is a function that given an $x, y$ position it will
return a probability of a obstacle being at $x, y$ within a certain amount of
time in the future. Using the function that is returned, we hypothesize that it
is possible to generate safe paths through cluttered dynamic environments where
there is a measurable uncertainty of an obstacle moving off its current path.

\section{Proposed Approach}

The current state of the art for planning around moving obstacles where the
trajectories are known consists of planning in space-time and by waiting until
the current path is clear, and afterwards moving towards the goal. The first
approach lacks the ability to deal elegantly with uncertainty in the
trajectories of the moving obstacles. The second approach generates suboptimal
paths in terms of the amount of time it takes to move from the initial
configuration to the goal configuration. Our approach will elegantly deal with
the uncertainty in the obstacles' trajectories whilst generating paths that
will lead the robot to the goal without waiting unless absolutely necessary. We
plan to do this using a Probabilistic Roadmap that will sample points in space
and time $(x, y, t)$. An edge will be added between a point $(x_0, y_0, t_0)$
and $(x_1, y_1, t_1)$ as long as it is feasible for the robot to move between
$(x_0, y_0)$ and $(x_1, y_1)$ in $t_1 - t_0$ seconds. The control model for the
robot will determine the feasibility of moving between two points. A PRM with a
suitable number of sample points will capture the space-time connectivity of
the environment.

Using this roadmap, we will be able to assign weights to nodes based on the
probability of an obstacle being in the same location as the node. The edges
will be weighted using a combination of the probability of a robot colliding
with an obstacle if travelling along said edge and the length of the edge.
Using a shortest path algorithm such as A*, we will be able to generate paths
in space-time that minimize the length of the path and the probability of
colliding with an obstacle. The path that is generated will be the initial path
used by the robot to plan its motions. Since obstacles may move off their
initial trajectories, the PRM can be used for replanning. Once it is noticed
that an obstacle has changed its trajectory, the roadmap can be re-weighted,
and a new shortest path can be generated from the current configuration of the
robot to the goal configuration.

% \bibliography{writeup} \bibliographystyle{ieeetr}

\end{document}
